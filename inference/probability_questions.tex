\documentclass[a4paper, 11pt]{article}
\usepackage{graphicx}
\usepackage{natbib}
\usepackage{dsfont}
\usepackage[left=3cm,top=3cm,right=3cm]{geometry}

\newcommand{\xx}{\mathbf{x}}	% The unknown parameters
\newcommand{\data}{\mathbf{D}}  % The data
\newcommand{\dx}{d^N\mathbf{x}} % Volume element in parameter space
\renewcommand{\topfraction}{0.85}
\renewcommand{\textfraction}{0.1}
\parindent=0cm

\title{Questions and Exercises\\
for Astro Hack Week 2015}
\author{Brendon J. Brewer}

\begin{document}
\maketitle

\section{Bayesian Inference}


\subsection{Bayes' Rule}
From the single-hypothesis version of Bayes' rule:
\begin{eqnarray}
P(H|D) &=& \frac{P(H)P(D|H)}{P(D)}
\end{eqnarray}
prove the version commonly used for a set of mutually exclusive and exhaustive
hypotheses $\{H_1, H_2, ..., H_N\}$:
\begin{eqnarray}
P(H_i|D) &=& \frac{P(H_i)P(D|H_i)}{P(D)}\label{eq:bayes}
\end{eqnarray}
for all $i \in \{1, 2, ..., N\}$.

\subsection{Discrete parameter estimation}
You are a customs agent. Among other things, you are supposed to prevent
drugs being smuggled in packages sent into the country. A colleague finds a
package containing two toys. Let the number of toys containing drugs be
$\eta$. Clearly the value of $\eta$ is either 0, 1, or 2. You drill into
one of the toys and find that it does not have drugs.
Calculate the posterior distribution for $\eta$ given that the tested
toy did not contain drugs. Assume a (discrete) uniform prior, i.e.
$P(\eta=0) = P(\eta=1) = P(\eta=2) = 1/3$.

{\tiny Hint: Use Bayes' rule in the form of Equation~\ref{eq:bayes}
three times, and note they all have the same denominator.}

\subsection{Continuous parameter estimation}
An X-ray source emits photons at a steady rate, but since it's so distant, we
only pick up a few photons per second.
A standard probabilistic model for the arrival times of the photons is the
``Poisson process''. A specific prediction of this model is that, if the
expected number of photons in a time interval
is $\mu$, the (discrete) probability distribution for the actual number of photons $x$ in that interval is a Poisson distribution:
\begin{eqnarray}
p(x | \mu) &=& \frac{\mu^x e^{-\mu}}{x!}
\end{eqnarray}
Find and plot the posterior distribution for $\mu$ given $x=5$. Use an improper
log-uniform prior proportional to $\mu^{-1}$.

\subsection{Prediction}
Use the posterior distribution obtained in the previous question to calculate
the predictive distribution for $x'$, the number of photons observed in a
different one second interval, given $x=5$. Plot the predictive distribution
and compare it with what you'd get by naively assuming the best fit
(maximum likelihood) value $\mu=5$ to make the prediction.\\

{\bf Hint 1: }Write down the probability distribution for $x'$ given
$\mu$ and $x$, and then marginalise out $\mu$.\\

{\bf Hint 2: }You may need to do some numerical integration.

\end{document}

